\documentclass{article}
\usepackage[utf8]{inputenc}
\usepackage[spanish]{babel}
\usepackage{listings}
\usepackage{graphicx}
\graphicspath{ {images/} }
\usepackage{cite}

\begin{document}

\begin{titlepage}
    \begin{center}
        \vspace*{1cm}
            
        \Huge
        \textbf{Calistenia parcial 1}
            
        \vspace{0.5cm}
        \LARGE
        Informatica 2
            
        \vspace{1.5cm}
            
        \textbf{Juan Keyner Fernández Hincapié}
            
        \vfill
            
        \vspace{0.8cm}
            
        \Large
        Despartamento de Ingeniería Electrónica y Telecomunicaciones\\
        Universidad de Antioquia\\
        Medellín\\
        Marzo de 2021
            
    \end{center}
\end{titlepage}

\tableofcontents
\newpage
\section{Sección introductoria}\label{intro}
Examinar la capacidad de dar instrucciones correctas a una computadora o persona.

\section{Sección de contenido} \label{contenido}
Estudiante de ingeniería de Telecomunicaciones de la Univercidad de Antioquia, grupo 7 de informatica 2 2021
 


\subsection{Agregar lista de instrucciones.}
%
A continuación, en el siguente enlace del video subido a youtube \citeonline{knuthwebsite}, https://youtu.be/jF-qkbaWWR0
\begin{lstlisting}[language=C++, label=codigo_ejemplo]
// Tarea desarrollada, compilado y ejecutado en overleaf.com//





\end{lstlisting}


\begin{enumerate}
\item  identificar que hay una hoja y dos tarjetas sobre la mesa. 
\item Con delicadeza, agarrar la hoja sin dañarla y desplazarla con   delicadeza
\item Poner con delicadeza, la hoja con una de sus caras sobre la mesa y soltar la hoja 
\item Agarrar las 2 tarjetas a la vez de tal manera que sus caras queden juntas. 

\item levantar las dos tarjetas a la vez y levantarlas 10 cm sobre  la mesa 

\item En el aire, ubicar las tarjetas de manera vertical. 

\item Separar las tarjetas del extremo inferior, tal que en el extremo superior haya un ángulo de 15 grados.

\item  desplazar las tarjetas con estas condiciones hacía la hoja.

\item Poner las tarjetas con delicadeza, con la figura hecha anteriormente sobre  la cara superior de la hoja y no soltar hasta que estén equilibradas.

\end{enumerate}






\bibliographystyle{IEEEtran}
\bibliography{references}




\end{document}




